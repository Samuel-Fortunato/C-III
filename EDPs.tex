\section{Equações derivadas parciais}

\subsection{Método de Características}
\subsubsection{Forma da equação}
Equações (quase) lineares de primeira ordem:
\begin{boxedeq}[0.3]
        a u_x + b u_y = c
\end{boxedeq}

\begin{itemize}
    \item \textbf{Função incógnita:} $u = u(x,y)$
    \item \textbf{Funções conhecidas:} $a = a(x,y) $, $b = b(x,y)$, $c = c(x,y,u)$
\end{itemize}

\subsubsection{Solução}
A solução é dada pela solução do sistema de EDOs:
\begin{align*}
    x'(s) &= a(x,y) \\
    y'(s) &= b(x,y) \\
    u'(s) &= c(x,y,u(x,y))
\end{align*}
ou equivalentemente:
\begin{boxedeq}
    ds = \frac{dx}{a} = \frac{dy}{b} = \frac{du}{c}
    \Leftrightarrow
    \begin{cases}
        \frac{dx}{a} = \frac{dy}{b} \\
        \frac{dx}{a} = \frac{du}{c}
    \end{cases}
\end{boxedeq}

Ao resolver o sistema, obtemos duas constantes de integração $C_1$ e $C_2$. Podemos obter a solução geral da EDP através da relação entre as constantes:
\begin{boxedeq}[0.2]
    C_2 = f(C_1)
\end{boxedeq}



\newpage



\subsection{Classificação de equações lineares de segunda ordem}
Forma geral da EDP linear de segunda ordem em 2D:
\begin{equation*}
        A u_{xx} + B u_{xy} + C u_{yy} + D u_x + E u_y + F u = G
\end{equation*}

Os termos de ordem superior são mais importantes para a natureza da equação, logo a classificação da EDP é feita através do discriminante:
\begin{boxedeq}[0.4]
    B(x,y)^2 - 4 A(x,y) C(x,y)
\end{boxedeq}

\begin{itemize}
    \item \fbox{$B^2 - 4AC > 0$} - equação \textbf{hiperbólica} \\
        \textit{ex:} equação da onda
    \item \fbox{$B^2 - 4AC = 0$} - equação \textbf{parabólica} \\
        \textit{ex:} equação do calor
    \item \fbox{$B^2 - 4AC < 0$} - equação \textbf{elíptica} \\
        \textit{ex:} equação de Laplace
\end{itemize}
$A$, $B$ e $C$ são funções, logo o \textbf{tipo pode mudar em função de $x$ e $y$}.



\subsection{Método de Fourier (separação de variáveis)}

Procuram se soluções da forma:
\begin{boxedeq}[0.3]
    u(x,y) = X(x) T(y)
\end{boxedeq}

Estas soluções correspondem à ideia de "ondas estacionárias": soluções cujo perfil X é constante no tempo mas com intensidade variável T.

Tendo como exemplo a equação do calor: $u_t = u_{xx}$, temos:
\begin{gather*}
    u_t = X(x) T'(t) \quad , \quad u_{xx} = X''(x) T(t) \\
    X(x) T'(t) = X''(x) T(t)
\end{gather*}
ou seja:
\begin{boxedeq}[0.3]
    \frac{T'(t)}{T(t)} = \frac{X''(x)}{X(x)} = -\lambda
\end{boxedeq}

Assim sendo, temos duas EDOs lineares das quais sabemos as soluções para $\lambda > 0$:
\begin{gather*}
    T'(t) = -\lambda T(t) \quad \Rightarrow \quad T(t) = C e^{-\lambda t} \\
    X''(x) = -\lambda X(x) \quad \Rightarrow \quad X(x) = A \cos(\sqrt{\lambda} x) + B \sin(\sqrt{\lambda} x)
\end{gather*}
logo obtemos duas soluções possíveis da equação:
\begin{boxedeq}[0.4]
    u(x,t) = C e^{-\lambda t} \cos(\sqrt{\lambda} x)
\end{boxedeq}
\begin{boxedeq}[0.4]
    u(x,t) = C e^{-\lambda t} \sin(\sqrt{\lambda} x)
\end{boxedeq}

Para escolhermos entre estas soluções, usamos as condições de fronteira.



\subsubsection{Condições de fronteira e iniciais}
Para obtermos soluções especificas precisamos de:
\begin{itemize}
    \item Uma condição inicial: $u(x,0) = f(x)$
    \item Condições de fronteira: valores de $u$ ou suas derivadas em $0$ e $L$.
\end{itemize}
As condições de fronteira mais comuns são:
\begin{itemize}
    \item \textbf{Condição de Dirichlet:} "arrefecimento perfeito", temperatura constante em 0 na fronteira
        \begin{boxedeq}[0.3]
            u(0,t) = u(L,t) = 0
        \end{boxedeq}
        Para verificar a condição nula nos extremos a função $f(x)$ deve ter apenas a componentente seno, com $\sqrt{\lambda} L = n \pi$, logo, se $f(x) = \sin(\frac{\pi n}{L} x)$, então:
        \begin{boxedeq}[0.3]
            \sqrt{\lambda} = \frac{n \pi}{L}, \quad n \in \mathbb{N}
        \end{boxedeq}
        \begin{boxedeq}[0.4]
            u(x,t) = e^{-(\frac{\pi n}{L})^2 t} \sin(\frac{\pi n}{L} x)
        \end{boxedeq}

    \item \textbf{Condição de Neumann:} "isolamento perfeito", fluxo de calor nulo na fronteira
        \begin{boxedeq}[0.3]
            u_x(0,t) = u_x(L,t) = 0
        \end{boxedeq}
        Para verificar a condição de derivada nula nos extremos a função $f(x)$ deve ter apenas a componentente cosseno, com $\sqrt{\lambda} L = n \pi$, logo, se $f(x) = \cos(\frac{\pi n}{L} x)$, então:
        \begin{boxedeq}[0.3]
            \sqrt{\lambda} = \frac{n \pi}{L}, \quad n \in \mathbb{N}_0
        \end{boxedeq}
        \begin{boxedeq}[0.4]
            u(x,t) = e^{-(\frac{\pi n}{L})^2 t} \cos(\frac{\pi n}{L} x)
        \end{boxedeq}
\end{itemize}



\subsubsection{Condição inicial com f(x) arbitrária}

\noindent\textbf{Condições de fronteira de Dirichlet ($u(0,t) = u(L,t) = 0$)}

Para uma condição inicial $u(x,0) = f(x)$ onde $f(x)$ é contínua (por partes) e admite a decomposição:
\begin{equation*}
    f(x) = \sum_{n=1}^{+\infty} \beta_n \sin(\frac{n \pi}{L} x)
\end{equation*}
onde
\begin{equation*}
    \beta_n = \frac{2}{L} \int_0^L f(x) \sin(\frac{n \pi}{L} x) dx
\end{equation*}
então a solução $u(x,t)$ é dada por:
\begin{boxedeq}
    u(x,t) = \sum_{n=1}^{\infty} \beta_n e^{-(\frac{n \pi}{L})^2 t} \sin(\frac{n \pi}{L} x)
\end{boxedeq}


\noindent\textbf{Condições de fronteira de Neumann ($u_x(0,t) = u_x(L,t) = 0$)}
Para condições de fronteira de Neumann, a decomposição de $f(x)$ é feita em cossenos:
\begin{equation*}
    f(x) = \frac{\alpha_0}{2} + \sum_{n=1}^{+\infty} \alpha_n \cos(\frac{n \pi}{L} x)
\end{equation*}
com
\begin{equation*}
    \alpha_n = \frac{2}{L} \int_0^L f(x) \cos(\frac{n \pi}{L} x) dx
\end{equation*}
e a solução $u(x,t)$ é dada por:
\begin{boxedeq}
    u(x,t) = \frac{\alpha_0}{2} + \sum_{n=1}^{\infty} \alpha_n e^{-(\frac{n \pi}{L})^2 t} \cos(\frac{n \pi}{L} x)
\end{boxedeq}
