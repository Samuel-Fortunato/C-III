\section{Integração múltipla}

\subsection{Integrais em \texorpdfstring{$\mathbb{R}^2$}{R\^2}}

Com $\mathbb{D} \in \mathbb{R}^2$ sendo o domínio de integração:
{\large
    \begin{equation*}
        \iint_{\mathbb{D}}f(x,y)\,dx\,dy
    \end{equation*}
}

\subsubsection{Tipos de domínio:}
\begin{itemize}
    \item \textbf{Tipo 1}:
        \subitem $\tcboxmath[colframe=red]{a<x<b} \qquad \wedge \qquad  \tcboxmath[colframe=blue]{f_1(x)<y<f_2(x)}$
        \subitem \begin{figure}[H]
            \centering
            \begin{tikzpicture}
                \begin{axis}[
                    axis lines=middle,
                    xmin=-2, xmax=1,
                    ymin=-1, ymax=1.5,
                    xlabel={$x$},
                    ylabel={$y$},
                ]
                    % Upper curve
                    \addplot[name path=upper1, blue, thick, domain=-2:1]
                        {e^x};
                    
                    % Lower curve
                    \addplot[name path=lower1, blue, thick, trig format=rad, domain=-2:1]
                        {sin(x)};
                    
                    % x = -1
                    \addplot[red]
                        ({-1},{x});
                    
                    % x = -1
                    \addplot[red]
                        ({0},{x});
    
                    % Shaded region
                    \addplot[
                        blue,
                        pattern=north east lines,
                        pattern color=blue,
                        opacity=0.5
                    ]
                        fill between[
                            of=upper1 and lower1,
                            soft clip={domain=-1:0}
                        ];
                \end{axis}
            \end{tikzpicture}
            \caption{Domínio de \textbf{tipo 1}}
        \end{figure}
    \item \textbf{Tipo 2}:
        \subitem $\tcboxmath[colframe=blue]{g_1(y)<x<g_2(y)} \qquad \wedge \qquad \tcboxmath[colframe=red]{c<y<d}$
        \subitem \begin{figure}[H]
            \centering
            \begin{tikzpicture}
                \begin{axis}[
                    axis lines=middle,
                    xmin=-0.5, xmax=1.5,
                    ymin=-0.5, ymax=1.1,
                    xlabel={$x$},
                    ylabel={$y$},
                ]
                    % Upper curve
                    \addplot[name path=upper2, blue, thick, domain=0:1.5]
                        {sqrt(x)};
                    
                    % Lower curve
                    \addplot[name path=lower2, blue, thick, domain=-0.5:1.5]
                        {x^2};
                    
                    % y = 0
                    \addplot[red]
                        {0};
                    
                    % y = 1
                    \addplot[red]
                        {1};
    
                    % Shaded region
                    \addplot[
                        blue,
                        pattern=north east lines,
                        pattern color=blue,
                        opacity=0.5
                    ]
                        fill between[
                            of=upper2 and lower2,
                            soft clip={domain=0:1}
                        ];
                \end{axis}
            \end{tikzpicture}
            \caption{Domínio de \textbf{tipo 2}}
        \end{figure}
\end{itemize}

\subsubsection{Cálculo do integral}

Seja $\mathbb{D}$ um domínio de \textbf{tipo 1}:
\begin{boxedeq}[0.7]
    \iint_{\mathbb{D}}f(x,y)\,dx\,dy \quad = \quad \int_{a}^{b}\left(\int_{f_1(x)}^{f_2(x)} f(x,y)\,dy\right)\,dx
\end{boxedeq}

Seja $\mathbb{D}$ um domínio de \textbf{tipo 2}:
\begin{boxedeq}[0.7]
    \iint_{\mathbb{D}}f(x,y)\,dx\,dy \quad = \quad \int_{c}^{d}\left(\int_{g_1(x)}^{g_2(x)} f(x,y)\,dx\right)\,dy
\end{boxedeq}

Se o domínio for de tipo 1 e tipo 2, os dois integrais \textbf{são iguais}.


\subsection{Integrais em \texorpdfstring{$\mathbb{R}^3$}{R\^3}}
Com domínio de integração $\mathbb{V} \in \mathbb{R}^3$:
{\Large
    \begin{equation*}
        \iint\int_{\mathbb{V}}f(x,y,z)\,dx\,dy\,dz
    \end{equation*}
}

\subsubsection{Tipos de domínio:}
\begin{itemize}
    \item \textbf{Tipo 1}:
        \subitem $f_1(x,y) < z < f_2(x,y) \qquad , \qquad \text{para todo} (x,y) \in \mathbb{B}$
        \subitem Onde $\mathbb{B}$ é um domínio de tipo 1 ou 2 em $\mathbb{R}^2$.
    \item \textbf{Tipo 2} (relativamente a $XoY$):
        \subitem $a<z<b \qquad \wedge \qquad (x,y) \in A(z)$
        \subitem Sendo $A(z)$ uma secção paralela ao plano $XoY$.
        
\end{itemize}

\subsubsection{Cálculo do integral}

Se $\mathbb{V}$ for um domínio de \textbf{tipo 1}:
\begin{boxedeq}[0.8]
    \iint\int_{\mathbb{V}}f(x,y,z)\,dx\,dy\,dz = \iint_\mathbb{B}\left(\int_{f_1(x,y)}^{f_2(x,y)} f(x,y,z)\,dz\right)\,dx\,dy
\end{boxedeq}

Se $\mathbb{V}$ for um domínio de \textbf{tipo 1}:
\begin{boxedeq}[0.8]
    \iint\int_{\mathbb{V}}f(x,y,z)\,dx\,dy\,dz = \int_a^b\left(\iint_{A(z)} f(x,y,z)\,dx\,dy\right)\,dz
\end{boxedeq}



\subsection{Mudança de coordenadas}
Dada uma mudança de coordenadas $(u,v) \rightarrow (x,y) = s(u,v)$, do domínio $\mathbb{D}_{u,v}^{*}$ para $\mathbb{D}_{x,y}$, o integral em $(x,y)$ pode ser calculado como:
\begin{boxedeq}[0.9]
    \iint_{\mathbb{D}_{x,y}} f(x,y)\,dx\,dy = \iint_{\mathbb{D}_{u,v}^{*}} f(x(u,v),y(u,v))\left|\det\left[\frac{\partial(x,y)}{\partial(u,v)}\right]\right|\,du\,dv
\end{boxedeq}

\subsubsection{Coordenadas polares}
A mudança de variável para coordenadas polares é dada por:
\begin{boxedeq}[0.4]
    s(r,\theta) = \begin{cases}
        x = r \cos(\theta) \\
        y = r \sin(\theta)
    \end{cases}
\end{boxedeq}

Ao mudar para coordenadas polares, o Jacobiano dá a mudança:
\begin{boxedeq}[0.3]
    dx\,dy = \tcboxmath[left=0pt, right=0pt, colframe=red!75!black]{r} \,dr\,d\theta
\end{boxedeq}

\subsubsection{Coordenadas cilíndricas}
A mudança de variável para coordenadas cilíndricas é dada por:
\begin{boxedeq}[0.4]
    s(r,\theta,z) = \begin{cases}
        x = r \cos(\theta) \\
        y = r \sin(\theta) \\
        z = z
    \end{cases}
\end{boxedeq}

Ao mudar para coordenadas cilindricas, o Jacobiano dá a mudança:
\begin{boxedeq}[0.4]
    dx\,dy\,dz = \tcboxmath[left=0pt, right=0pt, colframe=red!75!black]{r} \,dr\,d\theta\,dz
\end{boxedeq}

\subsubsection{Coordenadas esféricas}
A mudança de variável para coordenadas esféricas é dada por:
\begin{boxedeq}[0.5]
    s(R,\theta,\varphi) = \begin{cases}
        x = R \cos(\theta)\sin(\varphi) \\
        y = R \sin(\theta)\sin(\varphi) \\
        z = R \cos(\varphi)
    \end{cases}
\end{boxedeq}

Ao mudar para coordenadas esféricas, o Jacobiano dá a mudança:
\begin{boxedeq}[0.45]
    dx\,dy\,dz = \tcboxmath[left=0pt, right=0pt, colframe=red!75!black]{R^2\sin(\varphi)}\,dR\,d\theta\,d\varphi
\end{boxedeq}